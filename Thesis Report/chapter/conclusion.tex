\chapter{\chapterSix}
\label{chp:6}

\section{Conclusions}
This thesis starts by showing the potential of using SRAM PUF as a secure way to protect our key and data. Embraced with a bright prospect, it is unfortunate that the development of PUF in the real world seems to lack of public involvement. The currently available solution is usually locked to specific entities, such as companies or universities. There is no open source project available for tech enthusiast to embrace this amazing technology. Here, we initiate an open source project to develop software-based SRAM PUF technology using off-the-shelf SRAM.
% Moreover, the answer to the problem statement is summarized in few paragraphs below.

As mentioned in Chapter 1, this thesis' problem statement says "How to develop an open-source secure data and key storage scheme using off-the-shelf SRAM component and software-based SRAM PUF technology?". Based on the experiments, the answer of this question can be explained in few steps. The steps itself are quite similar with the one explained in Section \ref{ch:procedure_develop}.
First, get any off-the-shelf SRAM from the market. Then, locate the stable bits using data remanence algorithm. Afterward, test the stable bits on various conditions. If the stable bits show an acceptable result, continue use this SRAM. Otherwise, abandon and try another off-the-shelf SRAM. Later, using the enrollment scheme shown in Section \ref{ch:complete_enrollment_scheme},  generate the challenge and the helper data which can be utilized to reconstruct the PUF-generated key. Using this the PUF-generated key, use it as the root-of-trust in the secure data and key storage application. To secure data/key, we decided to encrypt it using AES-256 with a \textit{final key} as the encryption key. The final key itself is derived using HMAC-SHA3 256 as the KDF (key derivation function) by combining PUF-generated key and user password. We consider this scheme is secure because the only-way someone can access the encrypted data is by having the SRAM and knowing the user password. Thus, the data and the key protected by the scheme is still safe even though the PUF device is lost.
In addition, ensure all the library required in the constructed system is open-source.
Moreover, software-based SRAM PUF technology is achieved by using off-the-shelf SRAM (requires no added hardware design) and constructing the whole system using Arduino code and Python code.
% We also present an automated enrollment system. The system, inspired by the concept of master-slave, consists of an Arduino source code (act as a slave) and a python source code, perform as a master, run in PC.  Using this system, two types of SRAM are tested; Microchip 23LC1024 and Cypress CY62256NLL. Both are tested on the distribution of 0's and 1's in their cells, intra hamming distance, inter hamming distance and the effect of voltage and time interval between enrollment.

In problem statement section, there are also two goals defined. First is "to devise a secure data protection and key storage scheme based on SRAM PUF technology. The data and the key protected by the scheme has to be safe even though the PUF device is lost. Moreover, the scheme should work using off-the-shelf SRAM." This goal is achieved and the explanation can be found in previous paragraph.

From this first goal, several sub-questions are also arised. First question is "Can we build SRAM PUF using off-the-shelf SRAM?". The answer is yes, we can. We show that the stable bits of SRAM Cypress CY62256NLL which located using data remanence algorithm has a sufficient reliability to be a root-of-trust in SRAM PUF.

Second question is "If it is possible, what characteristics need to be fulfilled by off-the-shelf SRAM to be eligible as a PUF candidate?". Unfortunately, we cannot answer this question. As shown in Chapter 4, we also do experiment on another off-the-shelf SRAM called Microchip 23LC1024. This SRAM show an unsatisfying result to be a PUF candidate. We believe that the SRAM size and the technology used in SRAM manufacturing affects a lot of SRAM quality as a PUF candidate. For example, Cypress CY62256NLL has significantly larger size than Microchip 23LC1024 (a rough approximation results in 13.38 times larger). Cypress CY62256NLL also has a smaller capacity (256k) than Microchip 23LC1024 (1024k). In addition, Cypress CY62256NLL is produced using an older technology (90nm) compared to Microchip 23LC1024 which has a higher chance to be produced using a newer technology since it has a much smaller size but larger memory size than Cypress CY62256NLL (there is no information on manufacturing technology used in the production on their websites and the Microchip 23LC1024 manual descriptions). From these explanations, we can conclude that Cypress CY62256NLL has less density than Microchip 23LC1024. These reasons lead us to a confirmation of density effects  explained in Section \ref{ch:sram_noise} which says the more dense an SRAM, the more environments affect the performance of the SRAM. But does it mean that SRAM PUF cannot be produced using an SRAM with a high density? This seems untrue due to some SRAM PUF references mentioned a newer technology in their PUF constructions, e.g. Cortez et. al. in \cite{7102498} use SRAMs which produced using 32nm and 45nm (sadly, there is no information on the type and manufacturer of their tested SRAMs). Moreover, as mentioned in Section \ref{ch:prev_experiments}, the experiments done in \cite{Schrijen:2012:CAS:2492708.2493033} shown that an older manufacturing technology doesn't always produced a more stable SRAM (Cypress CY7C15632KV18 (65nm) is more stable than Virage HP ASAP SP ULP 32-bit (90nm), even though the most stable is IDT 71V416S15PHI which produced using 180nm technology).
Furthermore, since every company always has their own way of dealing with noises introduced by high density level, we cannot conclude that high density level always lead to low quality of an SRAM as a PUF candidate. This undefined conclusion lead us to other questions. Should we trust specific company such as Cypress and mistrust another company like Microchip? Or do we need to look into specific product to determine whether an SRAM is suitable for a PUF component? Should we always prefer SRAMs with less density? We suggest the communities and the academics to study this thing further.

The second goal defined in the problem statement is "create a sharing ecosystem for the evolution of our data protection and key storage scheme. The ecosystem should be easily accessed and understood to encourage the academics and commercial parties to use and develop the ecosystem together." We achieved this goal by providing all our codes with explanations on how to use them in a Github repository \cite{repository}.
%
% To answer the question whether we can build an SRAM PUF using off-the-shelf SRAM, we present testing results on two off-the-shelf SRAMs as SRAM PUF candidates; Microchip 23LC1024 and Cypress CY62256NLL. Both are tested on the distribution of 0's and 1's in their cells, intra hamming distance, inter hamming distance, and the effect of voltage variation and time interval between enrollment. Testing on two bit-selection algorithms (data remanence analysis and neighbor analysis) are also performed on both SRAMs. The testing results show that Cypress CY62256NLL is a qualified PUF candidate due to well distributed of 1's and 0's inside its memory and the stability of its stable bits produced by data remanence analysis against voltage variation and time variation between interval. This means that SRAM PUF is indeed can be built using off-the-shelf SRAM component.
% Sadly, SRAM Microchip 23LC1024 has displayed a poor performance to be eligible as a PUF candidate due to unbalanced 1's and 0's distribution and large HD\textsubscript{intra} when the stable bits are tested. We believe a factor that makes Cypress CY62256NLL performs better than Microchip 23LC1024 is its less density level. Unfortunately, we are not a hundred percent sure that high-density level always makes an off-the-shelf SRAM performs poorly as a PUF candidate due to some SRAM PUF references mentioned a newer technology in their PUF constructions, e.g. Cortez et. al. in \cite{7102498} use SRAMs which produced using 32nm and 45nm. Another reason is that every company always has their own way of dealing with noises introduced by high-density level, thus opening a possibility that maybe SRAM produced from a specific company is better than the other. We suggest the communities and the academics to study this problem further.
% These experiments also confirm the claim by Muqing et. al. in \cite{liu_zhou_tang_parhi_kim_2017} which says data remanence analysis is a better bit selection algorithm than neighbor analysis.
% Afterward, based on the testing results, we introduce a PUF enrollment scheme using data remanence analysis as the bit selection algorithm which will locate the location of the stable bits and SRAM Cypress CY62256NLL as the off-the-shelf SRAM component.


% We provide an enrollment mechanism to look for stable bits which will be used as the basis for PUF application. There are two algorithms for looking stable bits tested here; neighbors analysis and data remanence approach. Based on the experiment results, in our final enrollment scheme, we present a concrete SRAM PUF enrollment using an Arduino, an SRAM Cypress CY62256NLL and a microSD with data remanence approach as the bit selection algorithm. A microSD is required to store the helper data and the PUF challenge.

In addition, an idea to create numerous CRPs using SRAM PUF is also proposed here. Using a collection of bits as a challenge, the stable bits are permutated among themselves to create a challenge which has a tremendous number of possibilities.

Furthermore, we also introduce a secure data protection and key storage scheme using SRAM PUF. The proposed scheme is influenced by multi-factor authentication. Using a combination of a PUF-generated key and user's password, a derived key is produced and utilized as the final key to protect user's data or/and user's key. Unlike the automated enrollment system, this scheme only consists of an Arduino source code. Evaluation of this scheme time and code size are also presented.


\section{Future Work}

In this section, two major parts on possible future work are presented. The first part is about possible experiments to do on off-the-shelf SRAMs and the next part is related to possible developments on our secure data protection and key storage scheme. Explanations of these two will be provided below.

\subsubsection{Possible Experiments on Off-The-Shelf SRAMs}
In this thesis, the SRAM testing is only done on the effect of time interval between enrollment and voltage variation. We believe another testing on temperature and aging is required to ensure whether SRAM Cypress CY62256NLL is indeed a capable candidate for SRAM PUF. The capability to test on temperature and the aging effect is suggested to be included as an addition of our automated enrollment system.
In addition, we also encouraged others to test other types of SRAMs to enrich the knowledge of possible off-the-shelf SRAM as a PUF candidate and to check if a high-density level always leads to a poor performance for an SRAM to be a PUF candidate. Doing testing on other types of SRAMs can also confirm whether a product from specific is qualified as a PUF root-of-trust or not.


\subsubsection{Improvement on Secure Data Protection and Key Storage Scheme}
As mentioned in Section \ref{ch:testing_scheme}, during the time measurement of our proposed key storage scheme, two procedures which spend significant time is the reading challenge from microSD and the initialization stages. We suggest to further optimize these stages to give a better and faster performance.

This thesis only presents an idea to secure user's data using symmetric encryption. To see similar application but using asymmetric encryption concept, one should look further to the thesis done by Akhundov \cite{haji}. He presents a public key infrastructure (PKI) concept using the PUF-generated key as the root of trust. A possible integration between our work and his work is combining our 'final' key into his construction as a root of trust.

Moreover, our secure data protection and key storage scheme is only designed to work offline. We believe by making it works in an online scenario will lead to more usable applications in real life. The first step we suggest on evolving it to be an online scheme is by providing the Arduino with an internet connection and by storing the helper data and the challenge in the cloud infrastructure. This step will reduce the necessity for the Arduino to always connected to a microSD. To reconstruct the PUF-generated key, Arduino will just have to get the challenge and the helper data from the cloud. We also advise to do extensive security analysis if it is decided to work online since the risks in an online environment are numerous.

In addition, our idea of using user's password and the PUF-generated key is not the highest level of security in multi-factor authentication. As mentioned in Section \ref{chp:2.mfa}, the most secure multi-factor authentication can be achieved when all three factors are combined together; knowledge, possession, and inherence. Since there are only two factors utilize (knowledge and possession) in this thesis' proposed secure data protection and key storage scheme, an addition of inherence factor when generating the final key can increase the security level. As mentioned in Section \ref{ris_puf}, biometric-based authentication and PUF are utilized to secure self-sovereign identity in Pouwelse and de Vos's proposed technology stack during trust creation in blockchain era. A further read on their article mentioned that there is a working prototype of fingerprint authentication using a smartphone camera. Since that project and our work share the same principle, open source and open ecosystem, we suggest integrating this fingerprint authentication into our proposed scheme to enable an even higher level of security.
